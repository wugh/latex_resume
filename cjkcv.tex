%        File: cv.tex
%     Created: Mon Dec 08 08:00 PM 2014 C
% Last Change: Mon Dec 08 08:00 PM 2014 C
%
\documentclass[11pt,a4paper]{cjkresume}

\begin{document}

\name{吴国华}
\address{+86 185-0137-0591}
{ghmeta@163.com}
{男}
{北京市海淀区西土城路10号北京邮电大学学十1332}

\section{教育背景}
\noindent\begin{tabularx}{\textwidth}{@{}p{8em}p{10em}p{6em}p{16em}R@{}}
  \textbf{北京邮电大学} & 计算机科学与技术 & 工学硕士 & GPA:\@ 87.73/100 (学院排名4/400) & 2013.09\timedash{}2016.04 \\[0.3em]
\textbf{北京邮电大学} & 智能科学与技术 & 工学学士 & GPA:\@ 89.64/100 (专业排名1/32) & 2009.09\timedash{}2013.06
\end{tabularx}
\vspace{-1.7em}

%\education{北京邮电大学}
%{研究生在读}
%{计算机科学与技术}
%{2013.09\timedash{}至今}
%{GPA:\@ 87.73/100 (学院排名4/400)}

%\education{北京邮电大学}
%{学士学位}
%{智能科学与技术}
%{2009.09\timedash{}2013.06}
%{GPA:\@ 89.64/100 (专业排名1/32)}

\section{专业技能}
\begin{skilltable}
\textbf{编程语言:}    &     Python,C,C\verb|++|,Java\\
\textbf{专业知识:}     &    自然语言处理,对话管理,机器学习\\
\textbf{工具:}     &     Linux,Vim,Awk,Sed,Latex,Intellij Idea,Eclipse,Git,MySQL
\end{skilltable}

\section{实习经历}
\internship{阿里巴巴,阿里妈妈,钻石展位算法工程师}{2015.05\timedash{}2015.09}
\begin{workitemize}
  %通过添加PV质量过滤逻辑使得外投掘金定向的CTR提高24\%;钻展黑盒人群包离线效果评估工具开发。
  \item 钻展黑盒人群包项目:通过算法为广告主圈出其优质人群,降低广告主的投放门
    槛和操作成本。
    \begin{workdetailitemize}
      \item 实现基于用户淘宝内部行为的advertiser-based协同过滤算法帮助广告主圈出优
	质人群。
      \item 基于广告投放日志实现黑盒人群包离线效果评估工具开发。
    \end{workdetailitemize}
  \item 钻展广告投放效果监控工具开发:
    \begin{workdetailitemize}
      \item 通过开发钻展投放效果中间层报表,使得生成常用效果报表的时间由原来的30分
	钟缩短到1分钟。
      \item 钻展定向监控,产出各个定向整体的效果报表、广告位和广告主粒度的定向
	效果报表以及抄底流量成分报表。
    \end{workdetailitemize}
\end{workitemize}

\internship{Nuance,自然语言处理实习生}{2014.03\timedash{}2014.07}
\begin{workitemize}
  \item 参与开发车载语音助手中的自然语言理解模块,主要功能是对用户的文本命令进
    行分类和命令参数提取。
  \item 负责训练语料生成,通过收集到的模板和词典生成12个领域的文本命令,并对生
    成的文本命令进行预处理和特征抽取,最后生成最大熵模型和CRF模型的训练语料。
  \item 负责实现基于规则的文本命令分类和命令参数提取系统,在表达方式比较限定的
    领域进行测试,准确率达到95\%。
\end{workitemize}

\internship{百度,文库,测试开发实习生}{2012.10\timedash{}2013.04}
\begin{workitemize}
  \item 根据文库后端的API编写相应的自动化测试用例进行接口测试。
  \item 根据需求说明设计测试用例,对产品进行功能测试,Bug报告,编写测试报告。
  \item 编写Bash脚本进行自动化测试环境搭建。
\end{workitemize}

\section{项目经历}

\project{基于POMDP的口语对话系统实现~(团队人数: 1)}{主要开发人员}{2014.09\timedash{}至今}
\begin{workitemize}
  \item 以简单的教学任务为背景,实现一个基于Web的语音对话系统,把对话管理过程建
    模成部分可观测马尔可夫决策过程(POMDP)从语料中估计POMDP的相关参数,然后求
    解出POMDP的决策函数。对话过程中根据用户动作估计当前系统的状态分布,从而根据
    决策函数作出响应来控制对话过程。
  \item 实现基于最大熵和CRF的语义理解模块,基于POMDP的对话管理模块。
  \item 调用Google API实现语音识别和语音合成模块。
  \item 发表论文: Finite-to-Infinite N-Best POMDP for Spoken Dialogue Management。
\end{workitemize}

\project{SIGHAN中文切分~(团队人数: 2)}{主要开发人员}{2014.05\timedash{}2014.08}
\begin{workitemize}
  \item 参加SIGHAN-2014中文切分评测任务,实现基于CRF的中文切分系统,通过引入词
    典特征和无监督分词特征(Accessor Variety,AV)以及新词发现来增强模型的新词
    切分能力。
  \item 负责特征抽取和切分系统实现,通过添加词典和AV特征使得系统性能(F值)\emph{提高2\%}。
  \item \emph{获得评测任务第一名,准确率为0.968,召回率为0.978,F值为0.973}。
  \item 发表论文: Leveraging Rich Linguistic Features for Cross-domain Chinese Segmentation。
\end{workitemize}

\project{三星中文语料获取~(团队人数: 3)}{主要开发人员}{2013.09\timedash{}2013.12}
\begin{workitemize}
  \item 通过编写爬虫进行网页获取,对网页进行正文抽取,按照项目的需求对文本进行
    段落去重。
\end{workitemize}

\project{特定领域人际对话系统~(团队人数: 3)}{主要开发人员}{2012.03\timedash{}2012.07}
\begin{workitemize}
  \item 本系统在安卓平台上将LBS信息加入到特定领域的人机对话系统,使用自然对话的
    形式实现为用户查找推荐餐馆。
  \item 主要负责餐馆信息抽取、搭建数据库、信息检索、安卓App实现。
  \item \emph{获得全国大学生智能设计大赛二等奖}。
\end{workitemize}

\section{获奖情况}
\noindent\begin{minipage}{0.5\textwidth}
  \begin{workitemize}
    \item 2014.09\quad{}VMWARE企业奖学金~(排名4/400)
    \item 2012.09\quad{}长飞企业奖学金~(排名1/32)
    \item 2011.09\quad{}国家励志奖学金~(排名1/32)
  \end{workitemize}
\end{minipage}%
\hfill
\begin{minipage}{0.5\textwidth}
  \begin{workitemize}
    \item 2010.09\quad{}国家奖学金~(排名1/32)
    \item 2012.06\quad{}北京市大学生电子设计竞赛二等奖
    \item 2012.06\quad{}北京邮电大学创新项目一等奖
  \end{workitemize}
\end{minipage}%
%\scholarship{VMWARE企业奖学金}{年级排名第4获得}{2013.09\timedash{}2014.09}
%\scholarship{长飞企业奖学金}{专业排名第1获得}{2011.09\timedash{}2012.09}
%\scholarship{国家励志奖学金}{专业排名第1获得}{2010.09\timedash{}2011.09}
%\scholarship{国家奖学金}{专业排名第1获得}{2009.09\timedash{}2010.09}
%\award{北京市大学生电子设计竞赛二等奖}{2012.06}
%\award{北京邮电大学创新项目一等奖}{2012.06}

\end{document}


